\documentclass[11pt, a4paper]{article}
\usepackage{graphicx}
\usepackage{verbatim}
\title{ELEN4010 - Software Development Project \\ Coding Conventions}
\author{Kanaka Babshet (678851) and Ari Croock (718005)\\Daniel Weinberg (547937) and Alice Yang (597609)}
\date{2016/04/19}

\begin{document}
	\maketitle
	\tableofcontents
	\newpage
	\noindent
	\section{Introduction}
	Coding conventions are the methods and standards by which certain programs are coded. This document lists the coding conventions implemented for the development of a meeting scheduling system for the ELEN4010 Software Development project. 
\section{Python}
Version: Python 3.5.1
\subsection{Readability}
\begin{itemize}
	\item Tabs are used for indentation 
	\item Python3 disallows the mixed usage of tabs and spaces
	\item Continuation lines align wrapped elements vertically 
	\item The maximum line length is 79 characters, and the window width is 80 in order to avoid wrapping
	\item The line length of doc strings and comments are limited to 72 characters
	\item Long lines are wrapped by placing expressions in brackets
	\item Top-level function and class definitions are surrounded with two blank lines
	\item Method definitions inside a class are surrounded by a single blank line
	\item White space is only inserted after punctuation, and not before
	\item Lines should not end with semicolons
\end{itemize}

\subsection{Maintainability}

\begin{itemize}
	\item Class definitions are written using camelCase
	\item All other names are written using snake\_case
	\item Constants are defined in all capital letters
	\item There are no global variables
	\item \textit{self} is used for the first argument to instance methods
	\item Files are closed explicitly on program exit
	\item Imported scripts should not execute the core functionality of the code
	\item All code in the core implements the UTF-8 character encoding, without an encoding declaration 
	\item All imports must be done on individual lines and at the beginning of the script
\end{itemize}

\subsection{Documentation}
\begin{itemize}
	\item Short comments on one line succinctly describe functionality 
	\item Block comments are indented to the same level as the code that follows them
	\item Space must be inserted before the comment specifier
	\item Any comment that exists is kept up to date with changing code 
\end{itemize}
\section{HTML}
\subsection{Readability}
\begin{itemize}
	\item Indentations are tabbed
	\item New lines are used for every block
	\item Line lengths are limited to 80 characters
\end{itemize}
\subsection{Maintainability}
\begin{itemize}
	\item No style sheets are used to prevent complexity
	\item All code in the core implements the UTF-8 character encoding, without an encoding declaration
	\item  Clear and precise names for IDs and classes are used
	\item Names provided are lower case due to XML being case sensitive
	\item All non empty elements have a closing tag
	\item Consistent cases are used 
\end{itemize}
\subsection{Documentation}
\begin{itemize}
	\item Few comments are utilized to describe the HTML code
	\item The primary description relies on well defined variables, IDs and tags
	\item Short comments that exist are written on one line
	\item A space must be inserted before the comment specifier
	\item Any comment that exists is kept up to date with changing code 
\end{itemize}
\section{Javascript}
\subsection{Readability}
\begin{itemize}
	\item Javascript code is embedded in the HTML because the code is specific to a session
	\item Every statement begins in line vertically with the current indentation
	\item A space follows punctuation
	\item The maximum line length is 79 characters, and the window width is 80 in order to avoid wrapping
	\item Indentations are done using tabs

\end{itemize}
\subsection{Maintainability}
\begin{itemize}
	\item All variables and functions should be declared before use
	\item Naming is done using camel case
\end{itemize}
\subsection{Documentation}
\begin{itemize}
	\item Short comments that exist are written on one line
	\item A space must be inserted before the comment specifier
	\item Any comment that exists is kept up to date with changing code 
	\item Inline comments are preferred to block commenting
\end{itemize}

\section{SQL}
\subsection{Readability}
\begin{itemize}
	\item Tabs are avoided
	\item The maximum line length is 79 characters, and the window width is 80 in order to avoid wrapping
	\item Indents are 4 spaces for column names and table wide constraint definitions
	\item Only one statement is utilized on each line
	\item White space is implemented after punctuation
	\item Column definitions must be separated by 1 blank line
\end{itemize}
\subsection{Maintainability}
\begin{itemize}
	\item All SQL keywords are in upper case
	\item snake\_case is used for all function and variable names, and camelCase is not utilised
\end{itemize}
\subsection{Documentation}
\begin{itemize}
	\item All comments use the SQL standard double-dash method (- -)
	\item Does not make use of c-style commenting to enable cross platform support
	\item Column comments are used optionally only when they are useful
\end{itemize}

\section{Conclusion}
In conclusion, these coding conventions have the ultimate goal of improving the readability, maintainability and general understanding of the source code. It provides a standard for other programmers to grasp the algorithms used in order to improve the current code.


\end{document}